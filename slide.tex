\documentclass[slidestop,compress,mathserif,c]{beamer}
\usepackage{listings}
\usepackage{color}
\usepackage{xcolor}
\usepackage{hyperref}
\definecolor{dkgreen}{rgb}{0,0.6,0}
\usepackage{fontspec,xunicode,xltxtra,beamerthemesplit}
%\usepackage{slashbox}
\usepackage{diagbox}
 \newfontfamily\times{Times New Roman}
%以下是各种演示主题,定义幻灯片中的所有细节
%\usetheme{default}
%\usetheme{Berlin}%这个主题比较好
%\usetheme{Pittsburgh}
%\usetheme{Rochester}
\usetheme{Berkeley}%这种演示主题比较好
%\usetheme{Goettingen}
%\usetheme{Hannover}
%\usetheme{Marburg}
%\usetheme{PaloAlto}%这种演示主题比较好
%\usetheme{Antibes}
%\usetheme{Darmstadt}
%\usetheme{JuanLesPins}
%\usetheme{Montpellier}
%\usetheme{Singapore}
%\usetheme{Boadilla}
%\usetheme{Madrid}
%\usetheme{AnnArbor}
%\usetheme{CambridgeUS}
%\usetheme{Copenhagen}
%\usetheme{Warsaw}

%以下是各种外部主题,确定幻灯的显示式样
%\useoutertheme{infolines}
%\useoutertheme{miniframes}
%\useoutertheme[height=0.1\textwidth,width=0.15\textwidth,hideothersubsections]{sidebar}
%\useoutertheme{smoothbars}
%\useoutertheme{split}
%\useoutertheme{shadow}
%\useoutertheme{tree}
%\useoutertheme{smoothtree}
%\useoutertheme[height=0.5\textwidth]{sidebar}

%以下是各种内部主题
%\useinnertheme{default}
%\useinnertheme{circles}
%\useinnertheme{rectangles}
\useinnertheme[shadow]{rounded}

%以下是各种颜色主题
%\usecolortheme{default}
%\usecolortheme{albatross}
%\usecolortheme{beaver}
%\usecolortheme{beetle}
%\usecolortheme{crane}
%\usecolortheme{dolphin}
%\usecolortheme{dove}
%\usecolortheme{fly}
%\usecolortheme{lily}
%\usecolortheme{orchid}
%\usecolortheme{rose}
%\usecolortheme{seagull}
\usecolortheme{seahorse}
%\usecolortheme{sidebartab}
%\usecolortheme{structure}
%\usecolortheme{whale}
%\usecolortheme{wolverine}

%以下是各种字体主题
%\usefonttheme{default}
%\usefonttheme[onlymath]{serif}
%\usefonttheme{structurebold}
%\usefonttheme{structureitalicserif}
%\usefonttheme{structuresmallcapsserif}
\newcommand{\tablecell}[2]{\begin{tabular}{@{}#1@{}}#2\end{tabular}}   % 表格内换行,用法\tablecell{c}{balabalba\\balabalba}
\setsansfont[Mapping=tex-text, BoldFont={微软雅黑 Bold}]{微软雅黑}

% 中文环境自动换行
\XeTeXlinebreaklocale "zh"  % 表示用中文的断行
\XeTeXlinebreakskip = 0pt plus 1pt % 多一点调整的空间

% 中文环境修正导航栏
\makeatletter
\setbeamertemplate{blocks}[rounded][shadow=true] 
\def\beamer@linkspace#1{
  \begin{pgfpicture}{0pt}{-1.5pt}{#1}{5.5pt}
    \pgfsetfillopacity{0}
    \pgftext[x=0pt,y=-1.5pt]{.}
    \pgftext[x=#1,y=5.5pt]{.}
  \end{pgfpicture}}
\makeatother

% 超链接高亮显示
\hypersetup{CJKbookmarks=true,
colorlinks=true,
citecolor=blue,
linkcolor=blue,
urlcolor=blue,
bookmarksopen=true,
breaklinks=true
}

% 幻灯片切换方式
%\transblindshorizontal 	% 水平百叶窗
%\transblindsvertical 		% 垂直百叶窗
%\transboxin				% 盒状收缩
%\transboxout				% 盒状展开
%\transdissolve				% 溶解
%\transglitter
%\transsplithorizontalin	% 上下向中央收缩
%\transsplitverticalin		% 垂直向中央收缩
%\transsplithorizontalout	% 上下向中央展开
%\transsplitverticalout		% 垂直向中央展开
%\transwipe					% 从下抽出


\title{Gossip算法}
\author{Liu Zheng}
\date{\today}
\institute{同济大学电信学院}

\logo{\color{blue!50}\scalebox{2}{{\includegraphics[height=0.8cm]{logo.jpg}\vspace{220pt}}}}

% 设定frametitle居中
\makeatletter 
\long\def\beamer@@frametitle[#1]#2{% 
  \beamer@ifempty{#2}{}{% 
    \gdef\insertframetitle{\centering{#2\ifnum\beamer@autobreakcount>0\relax{}\space\usebeamertemplate*{frametitle continuation}\fi}}% 
  \gdef\beamer@frametitle{#2}% 
  \gdef\beamer@shortframetitle{#1}% 
}% 
} 
\makeatother


\begin{document}

\frame{\titlepage}

\section{Gossip背景}

%%%%%%%%%% One Slide %%%%%%%%%%
%\subsection{\hfill Overview}
\begin{frame}
\frametitle{Overview}
~~~~~~Gossip算法因为Cassandra而名声大噪,Gossip看似简单,但要真正弄清楚其本质远没看起来那么容易。为了寻求Gossip的本质,下面的内容主要参考Gossip的原始论文:《Efficient Reconciliation and Flow Control for Anti-Entropy Protocols》。
\end{frame}
%%%%%%%%%% Slide End %%%%%%%%%%

%%%%%%%%%% One Slide %%%%%%%%%%
\subsection{\hfill  Gossip背景}
\begin{frame}
\frametitle{Gossip背景}
~~~~~~Gossip算法如其名,灵感来自办公室八卦,只要一个人八卦一下,在有限的时间内所有的人都会知道该八卦的信息,这种方式也与病毒传播类似,因此Gossip有众多的别名“闲话算法”、“疫情传播算法”、“病毒感染算法”、“谣言传播算法”。

~~~~~~但Gossip并不是一个新东西,之前的泛洪查找、路由算法都归属于这个范畴,不同的是Gossip给这类算法提供了明确的语义、具体实施方法及收敛性证明。
\end{frame}
%%%%%%%%%% Slide End %%%%%%%%%%


%%%%%%%%%% One Slide %%%%%%%%%%
\subsection{\hfill 算法的发展历史}
\begin{frame}
\frametitle{算法的发展历史}
~~~~~~1972年Hajnal等人首次给出了Gossip问题(电话问题)的描述:有n个妇女,每个人都知道一条特有的流言,她们通过电话互相联系;任意两个妇女联系上以后,互相交流当前自己知道的所有流言;最少需要多少次联系,使得n个妇女每个人都知道所有流言?这使得对流言问题的研究正式登上历史舞台。
 
~~~~~~1972年Galton-Watson处理(简单分支处理)模型的出现,使得对Gossip的研究有了坚实的理论工具;在简单分支处理模型基础上,1975年Bailey对Gossip进行了更加详尽深入的理论分析;后来马尔科夫链成为Gossip算法分析的重要工具。总之Gossip算法的研究和分析有着丰富的理论工具:概率、分支处理、马尔科夫链等。
\end{frame}
%%%%%%%%%% Slide End %%%%%%%%%%

%%%%%%%%%% One Slide %%%%%%%%%%
%\subsection{\hfill 算法发展历史}
\begin{frame}
\frametitle{算法的发展历史}
 
~~~~~~1988年Hedetniem等人对计算机网络环境下的Gossip问题进行了精确的定义:A是由网络中所有节点组成的集合,每个节点都有自己特有的信息,并需将其传播到网络中其它所有节点;用有序的节点对$(i,j)i,j\in A$序列表示信息的传播过程,每个节点对表示两者之间存在信息交换;当序列最后一个节点对完成交互后,所有节点都知道所有信息,称为Gossip过程结束?需要多少次信息交换Gossip过程能够结束,需要多长时间?这是Gossip研究的核心内容。
 
\end{frame}
%%%%%%%%%% Slide End %%%%%%%%%%


\section{算法描述}

%%%%%%%%%% One Slide %%%%%%%%%%
%\subsection{\hfill hehe}
\begin{frame}
\frametitle{算法描述}
~~~~~~假设有 ${p, q, \cdots}$ 为协议参与者。 每个参与者都有关于一个自己信息的表。
用编程语言可以描述为: 
记 InfoMap = Map<Key, (Value, Version)>, 那么每个参与者要维护一个 InfoMap 类型的变量 localInfo。 同时每一个参与者要知道所有其他参与者的信息, 即要维护一个全局的表,即 Map<participant, InfoMap> 类型的变量 globalMap。每个参与者更新自己的 localInfo, 而由 Gossip 协议负责将更新的信息同步到整个网络上。
每个节点和系统中的某些节点成为 peer (如果系统的规模比较小,和系统中所有的其他节点成为 peer)。 
\end{frame}
%%%%%%%%%% Slide End %%%%%%%%%%


%%%%%%%%%% One Slide %%%%%%%%%%
%\subsection{\hfill hehe}
\begin{frame}
\frametitle{hehe}
 三种不同的同步信息的方法:
 \begin{itemize}
\item[1)]push-gossip: 最简单的情况下, 一个节点 p 向 q 发送整个 GlobalMap
\item[2)]pull-gossip: p 向 q 发送 digest, q 根据 digest 向 p 发送 p 过期的 (key, (value, version)) 列表
\item[3)]push-pull-gossip:与pull-gossip类似,只是多了一步,A再将本地比B新的数据推送给B,B更新本地 
 \end{itemize}
\end{frame}
%%%%%%%%%% Slide End %%%%%%%%%%


%%%%%%%%%% One Slide %%%%%%%%%%
\subsection{\hfill Gossip特点}
\begin{frame}
\frametitle{Gossip特点}
~~~~~~Gossip算法又被称为反熵(Anti-Entropy),熵是物理学上的一个概念,代表杂乱无章,而反熵就是在杂乱无章中寻求一致,这充分说明了Gossip的特点:在一个有界网络中,每个节点都随机地与其他节点通信,经过一番杂乱无章的通信,最终所有节点的状态都会达成一致。每个节点可能知道所有其他节点,也可能仅知道几个邻居节点,只要这些节可以通过网络连通,最终他们的状态都是一致的,当然这也是疫情传播的特点。

~~~~~~要注意到的一点是,即使有的节点因宕机而重启,有新节点加入,但经过一段时间后,这些节点的状态也会与其他节点达成一致,也就是说,Gossip天然具有分布式容错的优点。
\end{frame}
%%%%%%%%%% Slide End %%%%%%%%%%

%%%%%%%%%% One Slide %%%%%%%%%%
\subsection{\hfill Gossip本质}
\begin{frame}
\frametitle{Gossip本质}
~~~~~~Gossip是一个带冗余的容错算法,更进一步,Gossip是一个最终一致性算法。虽然无法保证在某个时刻所有节点状态一致,但可以保证在”最终“所有节点一致,”最终“是一个现实中存在,但理论上无法证明的时间点。

~~~~~~因为Gossip不要求节点知道所有其他节点,因此又具有去中心化的特点,节点之间完全对等,不需要任何的中心节点。实际上Gossip可以用于众多能接受“最终一致性”的领域:失败检测、路由同步、Pub/Sub、动态负载均衡。

~~~~~~但Gossip的缺点也很明显,冗余通信会对网路带宽、CUP资源造成很大的负载,而这些负载又受限于通信频率,该频率又影响着算法收敛的速度,后面我们会讲在各种场合下的优化方法。
\end{frame}
%%%%%%%%%% Slide End %%%%%%%%%%

%%%%%%%%%% One Slide %%%%%%%%%%
\subsection{\hfill  Gossip节点的通信方式及收敛性}
\begin{frame}
\frametitle{ Gossip节点的通信方式及收敛性}
 根据原论文,两个节点 A、B 之间存在三种通信方式:
 \begin{itemize}
 \item push: A节点将数据(key,value,version)及对应的版本号推送给B节点,B节点更新A中比自己新的数据
 \item pull:A仅将数据key,version推送给B,B将本地比A新的数据(Key,value,version)推送给A,A更新本地
 \item push/pull:与pull类似,只是多了一步,A再将本地比B新的数据推送给B,B更新本地
 \end{itemize}
\end{frame}
%%%%%%%%%% Slide End %%%%%%%%%%

%%%%%%%%%% One Slide %%%%%%%%%%
%\subsection{\hfill hehe}
\begin{frame}
\frametitle{hehe}
 ~~~~~~如果把两个节点数据同步一次定义为一个周期,则在一个周期内,push需通信1次,pull需2次,push/pull则需3次,从效果上来讲,push/pull最好,理论上一个周期内可以使两个节点完全一致。直观上也感觉,push/pull的收敛速度是最快的。

~~~~~~假设每个节点通信周期都能选择(感染)一个新节点,则Gossip算法退化为一个二分查找过程,每个周期构成一个平衡二叉树,收敛速度为$O(n^2)$,对应的时间开销则为$O(\log n)$。这也是Gossip理论上最优的收敛速度。但在实际情况中最优收敛速度是很难达到的,假设某个节点在第i个周期被感染的概率为$p_i$ ,第$i+1$个周期被感染的概率为$p_{i+1}$ ,则pull的方式:
~~~~~~~~~~~~~~~~$ p_{i+1} =p_i^2$

而push为:~~~~~~~~~~~~
$p_{i+1}=p_i(1-\frac{1}{n})^{n(1-p_i)}$
\end{frame}
%%%%%%%%%% Slide End %%%%%%%%%%

%%%%%%%%%% One Slide %%%%%%%%%%
%\subsection{\hfill hehe}
\begin{frame}
\frametitle{hehe}
 ~~~~~~显然pull的收敛速度大于push,而每个节点在每个周期被感染的概率都是固定的$p(0<p<1)$,因此Gossip算法是基于p的平方收敛,也成为概率收敛,这在众多的一致性算法中是非常独特的。
\end{frame}
%%%%%%%%%% Slide End %%%%%%%%%%

%%%%%%%%%% One Slide %%%%%%%%%%
%\subsection{\hfill hehe}
\begin{frame}
\frametitle{hehe}
 Gossip的节点的工作方式又分两种:
 \begin{itemize}
 \item Anti-Entropy(反熵):以固定的概率传播所有的数据
 \item Rumor-Mongering(谣言传播):仅传播新到达的数据
 \end{itemize}
 ~~~~~~Anti-Entropy模式有完全的容错性,但有较大的网络、CPU负载;Rumor-Mongering模式有较小的网络、CPU负载,但必须为数据定义”最新“的边界,并且难以保证完全容错,对失败重启且超过”最新“期限的节点,无法保证最终一致性,或需要引入额外的机制处理不一致性。我们后续着重讨论Anti-Entropy模式的优化。
\end{frame}
%%%%%%%%%% Slide End %%%%%%%%%%

%%%%%%%%%% One Slide %%%%%%%%%%
\subsection{\hfill Anti-Entropy的协调机制}
\begin{frame}
\frametitle{Anti-Entropy的协调机制}
~~~~~~ 协调机制是讨论在每次2个节点通信时,如何交换数据能达到最快的一致性,也即消除两个节点的不一致性。上面所讲的push、pull等是通信方式,协调是在通信方式下的数据交换机制。协调所面临的最大问题是,因为受限于网络负载,不可能每次都把一个节点上的数据发送给另外一个节点,也即每个Gossip的消息大小都有上限。在有限的空间上有效率地交换所有的消息是协调要解决的主要问题。
\end{frame}
%%%%%%%%%% Slide End %%%%%%%%%%

%%%%%%%%%% One Slide %%%%%%%%%%
%\subsection{\hfill hehe}
\begin{frame}
\frametitle{hehe}
 在讨论之前先声明几个概念:
 \begin{itemize}
 \item 令N = {p,q,s,...}为需要gossip通信的server集合,有界大小
 \item 令(p1,p2,...)是宿主在节点p上的数据,其中数据有(key,value,version)构成,q的规则与p类似。
 \end{itemize}
~~~~~~ 为了保证一致性,规定数据的value及version只有宿主节点才能修改,其他节点只能间接通过Gossip协议来请求数据对应的宿主节点修改。
\end{frame}
%%%%%%%%%% Slide End %%%%%%%%%%

%%%%%%%%%% One Slide %%%%%%%%%%
%\subsection{\hfill hehe}
\begin{frame}
\frametitle{精确协调(Precise Reconciliation)}
 ~~~~~~精确协调希望在每次通信周期内都非常准确地消除双方的不一致性,具体表现为相互发送对方需要更新的数据,因为每个节点都在并发与多个节点通信,理论上精确协调很难做到。精确协调需要给每个数据项独立地维护自己的version,在每次交互是把所有的(key,value,version)发送到目标进行比对,从而找出双方不同之处从而更新。但因为Gossip消息存在大小限制,因此每次选择发送哪些数据就成了问题。当然可以随机选择一部分数据,也可确定性的选择数据。对确定性的选择而言,可以有最老优先(根据版本)和最新优先两种,最老优先会优先更新版本最新的数据,而最新更新正好相反,这样会造成老数据始终得不到机会更新,也即饥饿。
当然,开发这也可根据业务场景构造自己的选择算法,但始终都无法避免消息量过多的问题。
\end{frame}
%%%%%%%%%% Slide End %%%%%%%%%%

%%%%%%%%%% One Slide %%%%%%%%%%
%\subsection{\hfill hehe}
\begin{frame}
\frametitle{整体协调(Scuttlebutt Reconciliation)}
~~~~~~ 整体协调与精确协调不同之处是,整体协调不是为每个数据都维护单独的版本号,而是为每个节点上的宿主数据维护统一的version。比如节点P会为(p1,p2,...)维护一个一致的全局version,相当于把所有的宿主数据看作一个整体,当与其他节点进行比较时,只需必须这些宿主数据的最高version,如果最高version相同说明这部分数据全部一致,否则再进行精确协调。
整体协调对数据的选择也有两种方法:
\begin{itemize}
\item 广度优先:根据整体version大小排序,也称为公平选择
\item 深度优先:根据包含数据多少的排序,也称为非公平选择。因为后者更有实用价值,所以原论文更鼓励后者
\end{itemize}
\end{frame}
%%%%%%%%%% Slide End %%%%%%%%%%

%%%%%%%%%% One Slide %%%%%%%%%%
\subsection{\hfill Cassandra中的实现}
\begin{frame}
\frametitle{Cassandra中的实现}
 经过验证,Cassandra实现了基于整体协调的push/push模式,有几个组件:
三条消息分别对应push/pull的三个阶段:
\begin{itemize}
\item GossipDigitsMessage
\item GossipDigitsAckMessage
\item GossipDigitsAck2Message
\end{itemize}
\end{frame}
%%%%%%%%%% Slide End %%%%%%%%%%

%%%%%%%%%% One Slide %%%%%%%%%%
%\subsection{\hfill hehe}
\begin{frame}
\frametitle{hehe}
 还有三种状态:
 \begin{itemize}
\item EndpointState:维护宿主数据的全局version,并封装了HeartBeat和\item ApplicationState
\item HeartBeat:心跳信息
\item ApplicationState:系统负载信息(磁盘使用率)
 \end{itemize}
\end{frame}
%%%%%%%%%% Slide End %%%%%%%%%%

%%%%%%%%%% One Slide %%%%%%%%%%
%\subsection{\hfill hehe}
\begin{frame}
\frametitle{hehe}
 Cassandra主要是使用Gossip完成三方面的功能:
\begin{itemize}
\item 失败检测
\item 动态负载均衡
\item 去中心化的弹性扩展
\end{itemize}
\end{frame}
%%%%%%%%%% Slide End %%%%%%%%%%

%%%%%%%%%% One Slide %%%%%%%%%%
\subsection{\hfill 总结}
\begin{frame}
\frametitle{总结}
~~~~~~ Gossip是一种去中心化、容错而又最终一致性的绝妙算法,其收敛性不但得到证明还具有指数级的收敛速度。使用Gossip的系统可以很容易的把Server扩展到更多的节点,满足弹性扩展轻而易举。
 
唯一的缺点是收敛是最终一致性,不使用那些强一致性的场景,比如2pc。
\end{frame}
%%%%%%%%%% Slide End %%%%%%%%%%

%%%%%%%%%% One Slide %%%%%%%%%%
%\subsection{\hfill hehe}
\begin{frame}
\frametitle{hehe}
 
\end{frame}
%%%%%%%%%% Slide End %%%%%%%%%%



%%%%%%%%%% One Slide %%%%%%%%%%
%\subsection{\hfill hehe}
\begin{frame}
\frametitle{hehe}
 
\end{frame}
%%%%%%%%%% Slide End %%%%%%%%%%


%%%%%%%%%% One Slide %%%%%%%%%%
%\subsection{\hfill hehe}
\begin{frame}
\frametitle{hehe}
 
\end{frame}
%%%%%%%%%% Slide End %%%%%%%%%%


%%%%%%%%%% One Slide %%%%%%%%%%
%\subsection{\hfill hehe}
\begin{frame}
\frametitle{hehe}
 
\end{frame}
%%%%%%%%%% Slide End %%%%%%%%%%




\end{document}
